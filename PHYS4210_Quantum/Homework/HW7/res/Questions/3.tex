\section{(5.17)}

\begin{parts}
\item This part is super straightforward, so I am guessing that the ``Explain your answers for each element'' is for the second part. We fill up the first $s$ orbital, then the $s$ orbital in the next energy level, then since the next energy level admits another value of $l$, we get a $p$ orbital. That's it.

  \begin{align*}
    \text{H:}& \ (1s) \\
    \text{He:}&\ (1s)^2 \\
    \text{Li:}&\ (1s)^2(2s) \\
    \text{Be:}&\ (1s)^2(2s)^2 \\
    \text{B:}&\ (1s)^2(2s)^2(2p) \\
    \text{C:}&\ (1s)^2(2s)^2(2p)^2 \\
    \text{N:}&\ (1s)^2(2s)^2(2p)^3 \\
    \text{O:}&\ (1s)^2(2s)^2(2p)^4 \\
    \text{F:}&\ (1s)^2(2s)^2(2p)^5 \\
    \text{Ne:}&\ (1s)^2(2s)^2(2p)^6 \\
  \end{align*}


\item The first four atoms in their ground states have $\ell=0$, meaning the letter will be $S$ for all of them. For hydrogren, the single electron can only have spin $s=1/2$, so $2S+1 = 2$. Therefore the grand total $J=1/2$, so Hydrogen has $^2 S_{1/2}$.

  Helium fills the $1s$ orbital, so they now occupy a singlet configuration with spin 0: thus Helium has $^1 S_0$.

  Lithium has a new electron in the $2s$ orbital. There is still no orbital angular momentum and the spin of a single filled $s$ orbital is zero so really this is the same as Hydrogen: $^2 S_{1/2}$.

  In a similar vein, Beryllium will be the same as Helium: $^1 S_0$.

  Boron fills both $(1s)$ and $(2s)$ orbitals. Again, these have 0 angular momentum, so all we really care about is the electron in the $(2p)$ orbital. It is a single electron, so it has spin $1/2$, and with orbital angular momentum $\ell=1$, the total angular momentum is either $3/2$ or $1/2$. The letter is now $P$ since $\ell=0$. So the two possibilities for Boron are: $^2P_{3/2}$ and $^2P_{1/2}$.

  For carbon, the two electrons can have total spin $1$ or $0$, and the total orbital angular momentum can now be 2, 1, or 0, so it's a bit more complicated. For $L=0$, it is simple, but when $L=1$, $S$ can be 0 or 1. In the latter, we therefore have $J=2,1,0$. Similarly, when $L=2$ and $S=1$, we will have $J=3,2,1$. So: $^1 S_0$, $^3 S_1$, $^1 P_1$, $^3 P_2$, $^3 P_1$, $^3 P_0$, $^1 D_2$, $^3 D_3$, $^3 D_2$, $^3 D_1$.

  Lastly, for nitrogen, $L=3,2,1$ or 0, and $S=3/2$ or $1/2$. Following a similar process as before we get: $^2 S_{1/2}$, $^4 S_{3/2}$, $^2 P_{3/2}$, $^2 P_{1/2}$, $^4 P_{5/2}$, $^4 P_{3/2}$, $^4 P_{1/2}$, $^2 D_{5/2}$, $^2 D_{3/2}$, $^4 D_{7/2}$, $^4 D_{5/2}$, $^4 D_{3/2}$, $^4 D_{1/2}$, $^2 F_{7/2}$, $^2 F_{5/2}$, $^4 F_{9/2}$, $^4 F_{7/2}$, $^4 F_{5/2}$, $^4 F_{3/2}$.
  
\end{parts}





%%% Local Variables:
%%% mode: LaTeX
%%% TeX-master: "../../HW7"
%%% End:
