\section{(3.18)}



For all of these, the last term involving the rate of change of the operator will be zero since none of the operators depend on time.


\begin{parts}
    


\item Starting with $Q=1$, the entire right-hand side is zero since $[\hat{H},1] = \hat{H} - \hat{H} = 0$ and 1 obviously has no time dependence, so

\begin{equation}
    \diff{}{t}(\braket{1}) = \diff{}{t}\br{\braket{\psi|\psi}} = 0.
\end{equation}

This is something we proved before - the normalization of the wave-function is independent of time, so we can normalize it at the most convenient time (usually $t=0$), and we are set forever.


\item For $\hat{Q}=\hat{H}$, it obviously commutes with itself and since it almost never has time dependence, this means that 

\begin{equation}
    \diff{}{t}\braket{H} = 0,
\end{equation}

which is just conservation of energy! Technically, it's that measurements of the energy on any given system are guaranteed to give back the same energy; it cannot change with time so it cannot somehow gain a different energy sometime later.



\item For $\hat{Q} = \hat{x}$, we first need to look at the commutator

\begin{equation}
    [\hat{H},x] = \br{\br{\frac{p^2}{2m} + V},x} = \frac{1}{2m}[p^2,x],
\end{equation}

since $x$ obviously commutes with $V(x)$. Now,

\begin{equation}
    [p^2,x] = p^2x - xp^2 = p \cdot px - xp \cdot p = p[p,x] - pxp - [x,p]p + pxp = -p[x,p] - [x,p]p = -2i\hbar p,
\end{equation}

so

\begin{equation}
    [\hat{H},x] = -\frac{i\hbar p}{m}.
\end{equation}

Plugging this in:

\begin{equation}
    \diff{\braket{x}}{t} = \frac{i}{\hbar}\Braket{\br{-\frac{i\hbar p}{m}}} = \frac{\braket{p}}{m},
\end{equation}

or

\begin{equation}
    m\diff{\braket{x}}{t} = \braket{p}.
\end{equation}

This is just what we had before when doing all the problems with determining the expectation values for various wavefunctions!




\item For $\hat{Q} = \hat{p}$, $[p^2,p] = p^3-p^3 = 0$ and (using a test function $f$)

\begin{equation}
    \left[ V, -i\hbar \diff{}{x} \right]f = -i\hbar V \diff{f}{x} + i\hbar\diff{}{x}(Vf) = i\hbar\diff{V}{x}.
\end{equation}

So, plugging everything in:

\begin{equation}
    \diff{\braket{p}}{t} = \frac{i}{\hbar}\br{-\hbar\diff{V}{x}} = -\Braket{\diff{V}{x}}.
\end{equation}

This is the same as Equation~(1.28) in Griffiths, which is \textbf{Ehrenfest's theorem}.





\end{parts}