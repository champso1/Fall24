\section{Intro}
\begin{itemize}
    \item From the double slit experiment, we can discover some crazy stuff from Quantum Mechanics:
    \begin{itemize}
        \item Using a normal (classical) gun, we don't find any interference in the resulting patterns, which is exactly what we would expect.
        \item Using a a wave gun, we do find interference, but this is also what we expect, since at the time, there was plenty of study on waves.
        \item If we use an electron gun, we also in fact find interference, but this is not what we expect. The electron-as-a-particle idea clearly cannot stand!
    \end{itemize}
    \item Normal wave intensity is given as the square of the height, which is a type of amplitude. In a similar vein, we can define probability distributions for our electrons as a more abstract amplitude:
        \begin{equation}
            P = \abs{\psi}^2.
        \end{equation}
    \item This is all we can know: the probaility of something happening (Schrodinger's Cat thought experiment). Only once we make a full measurement can we find out for sure that state of something.
    \item So, the electron, before measurement, is a wave, but after measurement, it is a particle.
\end{itemize}

%%% Local Variables:
%%% mode: LaTeX
%%% TeX-master: "../../Notes"
%%% End:
