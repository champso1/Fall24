\documentclass{article}
\usepackage{PreambleCommon}
\usepackage{minted}

\title{Assignment 5: Big-O Sorting \\[5pt] Part 1: Runtime \\[8pt] CS3305/W01 Data Structures}
\author{Casey Hampson}

\begin{document}
\maketitle


\section*{Solution}
\begin{enumerate}
\item The code:

\begin{minted}{java}

int sum = 0;
for (int i=0; i<n; i++) {
    sum++;
}

\end{minted}
    
will run at $\mathcal{O}(n)$ time, since each loop is just incrementing a variable, which runs at constant time.




\item We are considering the code:

\begin{minted}{java}

int sum = 0;
for (int i=0; i<n; i++) {
    for (int j=0; j<n; j++) {
        sum++;
    }
}

\end{minted}

For each iteration of the outer loop, the inner loop runs $n$ times, and each iteration of the inner loop runs in constant time (like we just found), so we have that
\begin{equation*}
    T(n) = c * n * n \rightarrow \mathcal{O}(n^2).
\end{equation*}






\item For the code

\begin{minted}{java}

int sum = 0;
for (int i=0; i<n; i++) {
    for (int j=0; j<n*n; j++) {
        sum++;
    }
}

\end{minted}

we have an almost identical case to the previous one, except that this inner loop runs $n*n$ times, so
\begin{equation*}
    T(n) = c*n*n*n \rightarrow \mathcal{O}(n^3).
\end{equation*}




\item We are considering the code:

\begin{minted}{java}

int sum = 0;
for (int i=0; i<n; i++) {
    for (int j=0; j<i; j++) {
        sum++;
    }
}

\end{minted}

This time, we can write out the series for some arbitrarily large $n$. Each iteration of the outer loop increases the iterations of the inner loop by 1, which itself runs at constant time (i'll just let this constant $c=1$), so we have
\begin{equation*}
    T(n) = 1 + 2 + 3 + \ldots + (n-1) = \frac{n(n-1)}{2} \rightarrow \mathcal{O}(n^2).
\end{equation*}






\item We are considering the code:

\begin{minted}{java}

int sum = 0;

for (int i = 0; i < n; i++) {
    for (int j = 0; j < i * i; j++) {
        for (int k = 0; k < j; k++) {
            sum++;
        }
    }
}

\end{minted}

We need to look at this one a little more closely. For some value of $i$ during one of the $n$ iterations of the outer loop, the middle loop will then run $i^2$ times, so the innermost loop does its full loop that many times, where each time it does its full loop, its number of iterations increases up from 1 to $i^2-1$. This particular series looks like:
\begin{equation*}
    T(i) = 1 + 2 + 3 + \ldots + (i^2-1) = \frac{i^2(i^2-1)}{2} \sim i^4.
\end{equation*}
So, for each value of $i$ or each iteration of the outermost loop, we increment our variable (on the order of) $i^4$ times. The full series looks like
\begin{equation*}
    \sum_{n=0}^{n-1} i^4.
\end{equation*}
I'm honestly not entirely sure how we are meant to approach this, since the form of this sum isn't similar to any of the cases presented in the book, but as a physicist my intuition is to take the limiting case for large $n$, where the sum turns into an integral. Since we basically are just tossing away constants, it makes things really easy:
\begin{equation*}
    \lim_{n\rightarrow\infty} \sum_{n=0}^{n-1} i^4 \sim \int_0^n \dd i \;i^4 \sim n^5,
\end{equation*}
So, this particular code has $\mathcal{O}(n^5)$.
    
\end{enumerate}


\end{document}
