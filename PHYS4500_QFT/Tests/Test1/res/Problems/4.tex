\section{}

We are given a Lagrangian for a scalar field:

\begin{equation}
    \lag = \frac{1}{2}\ddp^{\mu}\phi\ddp_{\mu}\phi - \frac{1}{2}m^2\phi^2 - \frac{\lambda}{4!}\phi^4
\end{equation}

where $\lambda$ is some constant.

The first term in the Euler-Lagrange equation

\begin{equation*}
    \ddp_{\mu}\left( \diffp[]{\lag}{(\ddp_{\mu}\phi)} \right) = \ddp_{\mu} \left( \ddp^{\mu}\phi \right) = \ddp_{\mu}\ddp^{\mu}\phi.
\end{equation*}

The second term is:

\begin{equation*}
    \diffp[]{\lag}{\phi} = -m^2\phi - \frac{\lambda}{6}\phi^3.
\end{equation*}

So the solution to the Euler-Lagrange equation is:

\begin{equation*}
    \ddp_{\mu}\ddp^{\mu}\phi + m^2\phi + \frac{\lambda}{6}\phi^3 = 0.
\end{equation*}

The conjugate momentum can be found by

\begin{equation}
    \pi(x) = \diffp[]{\lag}{(\ddp_0\phi)}.
\end{equation}

We can rewrite the Lagrangian to make this a little more clear:

\begin{equation*}
    \lag = \frac{1}{2}\ddp_0\phi\ddp^0\phi + \frac{1}{2}\ddp_i\phi\ddp^i\phi - \frac{1}{2}m^2\phi^2 - \frac{\lambda}{4!}\phi^4,
\end{equation*}

so now the conjugate momentum is simply:

\begin{equation*}
    \pi(x) = \ddp^0\phi = \dot{\phi}.
\end{equation*}

The stress energy tensor is given by

\begin{align}
    T^{\mu\nu} &= \diffp[]{\lag}{(\ddp_{\mu})}\ddp^{\nu}\phi - g^{\mu\nu}\lag \\
    &= \ddp^{\mu}\phi\ddp^{\nu}\phi - g^{\mu\nu}\left( \frac{1}{2}\ddp_{\rho}\phi\ddp^{\rho}\phi - \frac{1}{2}m^2\phi^2 - \frac{\lambda}{4!}\phi^4 \right).
\end{align}

We have to use different indices for the 4-gradients inside the parentheses, since they are meant to be contracted only with each other, not the metric outside.

The Hamiltonian density $\ham$ is given with $T^{00}$, so

\begin{align*}
    \ham &= T^{00} = \ddp^0\phi\ddp^0\phi - \frac{1}{2}\left(\ddp_{\mu}\phi\ddp^{\mu}\phi - m^2\phi^2 - \frac{2\lambda}{4!}\phi^4\right), \\
    &= (\ddp_0\phi)^2 - \frac{1}{2}\left( \ddp_0\phi\ddp^0\phi + \ddp_i\phi\ddp^i\phi - m^2\phi^2 - \frac{2\lambda}{4!}\phi^4 \right), \\
    &= (\ddp_0\phi)^2 - \frac{1}{2}\left( (\ddp_0)^2 - (\grad\phi)^2 - m^2\phi^2 - \frac{2\lambda}{4!}\phi^4 \right), \\
    &= \frac{1}{2} \left[ (\ddp_0)^2 + (\grad\phi)^2 + m^2\phi^2 + \frac{2\lambda}{4!}\phi^4  \right].
\end{align*}

As expected, this is the same as the Hamiltonian energy density for the free real KG field plus an extra $\phi^4$ term. Additionally, it is also positive-definite, which is a good sign.