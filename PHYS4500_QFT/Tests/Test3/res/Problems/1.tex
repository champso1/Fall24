\section{}

The gauge transformation that the gauge fields must satisfy in Yang-Mills theories is

\begin{equation}
  \label{eq:1}
  \vv{\sigma}\cdot\vv{A}_{\mu} \rightarrow U\vv{\sigma}\cdot\vv{A}_{\mu}U^{-1} + \frac{i}{q} (\ddp_{\mu}U) U^{-1}.
\end{equation}

If we assume a small $\lambda$, then we can express an arbitary $SU(2)$ transformation like

\begin{equation}
  U = e^{iq\vv{\sigma}\cdot\vv{\lambda}} \rightarrow 1 + iq\vv{\sigma}\cdot\vv{\lambda}.
\end{equation}

Additionally, as an element of $SU(2)$, $U$ is unitary, meaning

\begin{equation}
  U^{-1} = U^{\dagger} = 1 - iq\vv{\sigma}\cdot\vv{\lambda}.
\end{equation}

Now, we just replace all occurrenced of $U$ in Equation~\eqref{eq:1} with the small $\lambda$ version:

\begin{equation}
  \vv{\sigma}\cdot{\vv{A}_{\mu}} \rightarrow (1 + iq\vv{\sigma}\cdot\vv{\lambda}) \vv{\sigma}\cdot\vv{\lambda} (1 - iq\vv{\sigma}\cdot\vv{\lambda}) + \frac{i}{q}[\ddp_{\mu}(1 + iq\vv{\sigma}\cdot\vv{\lambda})](1 -iq\vv{\sigma}\cdot\vv{\lambda})
\end{equation}

Looking at just the first term, we have

\begin{equation}
  \rightarrow iq\vv{\sigma}\cdot\vv{\lambda} + iq(\vv{\sigma}\cdot\vv{\lambda})(\vv{\sigma}\cdot\vv{A}_{\mu}) -iq(\vv{\sigma}\cdot\vv{A}_{\mu})(\vv{\sigma}\cdot\vv{\lambda}) + q^2(\vv{\sigma}\cdot\vv{\lambda})(\vv{\sigma}\cdot\vv{\lambda}).
\end{equation}

Based on the hint given in the problem, we know that the last term will be zero since it involves a cross product of the same vector ($\vv{\lambda}$) with itself. Thus,

\begin{equation}
  \rightarrow iq\vv{\sigma}\cdot\vv{\lambda} + iq(\vv{\lambda}\cdot\vv{A}_{\mu} + i\vv{\sigma}\cdot\vv{\lambda}\times\vv{A}_{\mu}) -iq(\vv{\lambda}\cdot\vv{A}_{\mu} + i\vv{\sigma}\cdot\vv{A}_{\mu}\times\vv{\lambda}).
\end{equation}

The dot product cancels, and we can switch the order of the cross product at the cost of a minus sign so that

\begin{equation}
  \rightarrow \vv{\sigma}\cdot\vv{A}_{\mu} - 2q\vv{\sigma} \cdot \vv{\lambda}\times\vv{A}_{\mu} \rightarrow \vv{\sigma} \cdot (\vv{A}_{\mu} - 2q\vv{\lambda}\times\vv{A}_{\mu}).
\end{equation}

For the second term, the derivative will only act on the $\vv{\lambda}$:

\begin{align}
  \rightarrow& \frac{i}{q}[\ddp_{\mu}(1 + iq\vv{\sigma}\cdot\vv{\lambda})](1 - iq\vv{\sigma}\cdot\vv{\lambda}) \\
  \rightarrow& \frac{i}{q}[iq \vv{\sigma}\cdot(\ddp_{\mu}\vv{\lambda})](1 - iq\vv{\sigma}\cdot\vv{\lambda}) \\
  \rightarrow& -\vv{\sigma} \cdot (\ddp_{\mu}\vv{\lambda}) + [\vv{\sigma} \cdot (\ddp_{\mu}\vv{\lambda})(\vv{\sigma}\cdot\vv{\lambda})].
\end{align}

Just as before, the last term sill be zero. So, putting it all together and pulling out a $\vv{\sigma}$:

\begin{equation}
  \vv{\sigma}\cdot\vv{A}_{\mu} \rightarrow \vv{\sigma} \cdot [\vv{A}_{\mu} - \ddp_{\mu}\vv{\lambda} - 2q\vv{\lambda} \times \vv{A}_{\mu}].
\end{equation}

Since we now have pulled out a $\vv{\sigma}$ on both sides, we can consider only the transformation on the field:

\begin{equation}
  \boxed{\vv{A}_{\mu} \rightarrow \vv{A}_{\mu} - \ddp_{\mu}\vv{\lambda} - 2q \vv{\lambda}\times\vv{A}_{\mu}.}
\end{equation}



%%% Local Variables:
%%% mode: LaTeX
%%% TeX-master: "../../Test3"
%%% End:
