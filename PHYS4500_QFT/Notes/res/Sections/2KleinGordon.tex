\section{The Klein Gordon Equation}


Now to some more cool/new stuff.


\begin{itemize}
    \item The Schr\"odiger Equation (SE) did very well for atomic/non-relativistic physics, where speeds weren't necessarily all that fast:
        \begin{equation}
            \hat{H}\Psi = i\hbar \diffp{\Psi}{t},
        \end{equation}
        where $\hat{H}$ is the Hamiltonian operator
        \begin{equation}
            \hat{H} = -\frac{\hbar}{2m} \vv{\nabla}^2 + V.
        \end{equation}
    \item Here, we have used the classical/Newtonian Hamiltonian but applied the quantum prescription that $p \rightarrow -i\hbar \grad$ and let them become operators that act on a wave function.
    \item Additionally, there is the prescription that $E \rightarrow i\hbar \diffp{}{t}$.
    \item These are motivated (but not necessarily derived; I believe the correct terminology for the derivation has to do with canonical quantization or something) by Noether's theorem and the idea of symmetries.
    \item If there is space-translational symmetry, it implies conservation of linear momentum, which is why we have a spatial derivative taking the place of the momentum.
    \item Similarly, for time-translational symmetry, we have conservation of energy, which is why we have a time derivative taking the place of the energy.
\end{itemize}


\begin{itemize}
    \item However, the point of all of this is that these are quantum prescriptions, which are good, but they are only prescriptions on the inherently non-relativistic Newtonian version of the Hamiltonian.
    \item Additionally, the SE is manifestly not relativistic due to its different treatment of space and time by having different orders in their derivatives.
    \item This just won't work.
\end{itemize}


\begin{itemize}
    \item So let's try and fix it by first just starting with a free particle, meaning there is no potential energy function.
    \item Instead of using the non-relativistic Hamiltonian, let's use the relativistic energy-momentum relation $p_{\mu}p^{\mu} = m^2$.
    \item Additionally, let's change the prescriptions to be relativistic, since before, specifically the momentum one, just resulted in the classical gradient. This new prescription will be $p^{\mu} \rightarrow i\hbar \partial^{\mu}$. 
    \item By doing this, plugging into our relation, and letting it act on some wave function $\phi$ (a different letter for differentiating between this and the non-relativistic version):
        \begin{gather}
            p_{\mu}p^{\mu}\phi = m^2\phi \\
            -\partial_{\mu}\partial^{\mu} \phi = m^2 \phi \\
            \boxed{(\partial_{\mu}\partial^{\mu} + m^2)\phi = 0}
        \end{gather}
    \item This is the \textbf{Klein-Gordon Equation}. Interestingly, Schr\"odinger came up with this before his SE, but since it didn't work for the hydrogen atom (due to spin), he abandoned it. Klein and Gordon then, a little while later, independently came up with this equation again and published anyway before spin (and the other problems) were analyzed.
    \item The solutions to this equation are plane-wave solutions:
        \begin{equation}
            \phi(x^{\mu}) = A\exp\left(-\frac{i}{\hbar} p_{\mu}x^{\mu}\right)\label{KleinGordonSolutions}
        \end{equation}
\end{itemize}



\begin{itemize}
    \item Now, from the non-rel. case, we had that for a wavefunction $\psi$ that obeyed the SE, the Born interpretation meant that $\abs{\psi}^2 = \psi^*\psi$ gives the probability density. This value is always positive definite, which is a good thing.
    \item On the other hand, $\abs{\phi}^2$ is not always positive. This is one of the reasons Schr\"odinger held back on publishing this equation as well, because this was not understood at the time. 
\end{itemize}




\begin{itemize}
    \item Back again to the SE, we had that the wave function satisfied a \textit{probability current}
        \begin{equation}
            \vv{j} = -\frac{i\hbar}{2m} \left(\psi^* \grad\psi - \psi\grad\psi^*\right),
        \end{equation}
        which is analogous to the electric current, in the sense that it must be conserved, i.e. it satisfies some \textit{continuity equation}:
        \begin{equation}
            \diffp{\psi}{t} + \grad \cdot \vv{j} = 0.
        \end{equation}
    \item To make this relativistic so it works in our theory, we need to consider a temporal component in some way. So, let's define
        \begin{equation}
            \rho \equiv -\frac{i\hbar}{2m} \left(\phi^* \diffp{\phi}{t} - \phi\diffp{\phi^*}{t}\right).
        \end{equation}
    \item We can now define a new 4-vector current
        \begin{equation}
            j^{\mu} \equiv (\rho, \vv{j}) = \frac{i\hbar}{2m}\left(\phi^*\ddp^{\mu}\phi - \phi\ddp^{\mu}\phi^*\right),
        \end{equation}
    \item Which satisfies the continuity equation
        \begin{equation}
            \ddp_{\mu}j^{\mu} = 0.
        \end{equation}
    \item Let's show that this works:
        \begin{align}
            \ddp_{\mu}j^{\mu} &= \ddp_{\mu} \left[\frac{i\hbar}{2m}\left(\phi^*\ddp^{\mu}\phi - \phi\ddp^{\mu}\phi^*\right)\right] \\
            &= \frac{i\hbar}{2m} \left[\ddp_{\mu}\phi^*\ddp^{\mu}\phi + \phi^*\ddp_{\mu}\ddp^{\mu}\phi - \ddp_{\mu}\phi\ddp^{\mu}\phi^* - \phi\ddp_{\mu}\ddp^{\mu}\phi^*\right].
        \end{align}
        The first and third terms cancel since they are the same, and by using the Dirac equation and its complex conjugate, this becomes
        \begin{equation}
            \ddp_{\mu}j^{\mu} = \frac{i\hbar}{2m} \left[-\phi^* m^2 \phi + \phi m^2 \phi^*\right] = 0.
        \end{equation}
        
\end{itemize}