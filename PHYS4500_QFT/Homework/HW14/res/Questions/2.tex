\section{}

\begin{align}
  \lambda^5\lambda^8 &= \frac{1}{\sqrt{3}} \begin{pmatrix}0 & 0 & -i \\ 0 & 0 & 0 \\ i & 0 & 0\end{pmatrix}\begin{pmatrix}1 & 0 & 0 \\ 0 & 1 & 0 \\ 0 & 0 & -2\end{pmatrix} \\
  &= \frac{1}{\sqrt{3}}\begin{pmatrix}0 & x & x \\ x & 0 & x \\ x & x & 0\end{pmatrix},
\end{align}

where, in the context of the trace, we don't care about the non-diagonal components, so I just labeled them with an $x$ and didn't bother computing them. We can see quite clearly that since all the diagonal elements are zero that

\begin{equation}
  \boxed{\Tr[\lambda^5\lambda^8] = 0.}
\end{equation}

We expect this, since

\begin{equation}
  \Tr[\lambda^i,\lambda^j] = 2\delta_{ij} = 0
\end{equation}

for $i \neq j$.








%%% Local Variables:
%%% mode: LaTeX
%%% TeX-master: "../../HW14"
%%% End:
