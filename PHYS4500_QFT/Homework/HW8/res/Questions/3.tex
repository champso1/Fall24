\section{(14.2)}

If the decay rate for a particle is $\Gamma = \qty{10}{\mega\electronvolt} = \qty{e-2}{\giga\electronvolt}$, then its average lifetime is given by the inverse of the decay rate:

\begin{equation}
    \tau = \frac{1}{\Gamma} = \qty{e2}{\per\giga\electronvolt}.
\end{equation}

This is in natural units. We know that from our prescription of $\hbar = 1$, this means that we have

\begin{equation*}
    \qty{6.582e-25}{\giga\electronvolt\second} = 1, \quad \mathrm{or} \quad \qty{1}{\per\giga\electronvolt} = \qty{6.582e-25}{s},
\end{equation*}

where we have now related seconds and inverse energy. Now, we can say that

\begin{equation*}
    \tau = \qty{e2}{\per\giga\electronvolt} = \boxed{\qty{6.582e-23}{s}.}    
\end{equation*}

Next, the half-life is given by $t_{1/2} = \tau \ln2$, so all we need to do is multiply by $\ln2$:

\begin{equation*}
    t_{1/2} = \ln2 \times \qty{6.582e-23}{s} = \boxed{\qty{4.562e-23}{s}.}
\end{equation*}