% Draw the NLO diagrams for e^- + e^+ -> u^- + u^+
\section{}

There are four real-emission diagrams since any of the four external particles can emit a photon.


\begin{center}
\begin{tikzpicture}
\begin{feynman}
    \vertex (a);
    \vertex[right of=a] (b);
    \vertex[above left of=a] (i1) {$e^-$};
    \vertex[below left of=a] (i2) {$e^-$};
    \vertex[above right of=b] (f1) {$\mu^-$};
    \vertex[below right of=b] (f2) {$\mu^+$};

    \vertex[above left=0.75cm of a] (x1);
    \vertex[above right=0.75cm of x1] (x2);

    \diagram* {
        (i1) --[fermion] (a) --[photon] (b) --[fermion] (f1),
        (i2) --[anti fermion] (a), (b) --[anti fermion] (f2),
        (x1) --[photon] (x2)
    };
\end{feynman}
\end{tikzpicture}
\begin{tikzpicture}
\begin{feynman}
    \vertex (a);
    \vertex[right of=a] (b);
    \vertex[above left of=a] (i1) {$e^-$};
    \vertex[below left of=a] (i2) {$e^-$};
    \vertex[above right of=b] (f1) {$\mu^-$};
    \vertex[below right of=b] (f2) {$\mu^+$};

    \vertex[below left=0.75cm of a] (x1);
    \vertex[below right=0.75cm of x1] (x2);

    \diagram* {
        (i1) --[fermion] (a) --[photon] (b) --[fermion] (f1),
        (i2) --[anti fermion] (a), (b) --[anti fermion] (f2),
        (x1) --[photon] (x2)
    };
\end{feynman}
\end{tikzpicture}

\begin{tikzpicture}
\begin{feynman}
    \vertex (a);
    \vertex[right of=a] (b);
    \vertex[above left of=a] (i1) {$e^-$};
    \vertex[below left of=a] (i2) {$e^-$};
    \vertex[above right of=b] (f1) {$\mu^-$};
    \vertex[below right of=b] (f2) {$\mu^+$};

    \vertex[above right=0.75cm of b] (x1);
    \vertex[above left=0.75cm of x1] (x2);

    \diagram* {
        (i1) --[fermion] (a) --[photon] (b) --[fermion] (f1),
        (i2) --[anti fermion] (a), (b) --[anti fermion] (f2),
        (x1) --[photon] (x2)
    };
\end{feynman}
\end{tikzpicture}
\begin{tikzpicture}
\begin{feynman}
    \vertex (a);
    \vertex[right of=a] (b);
    \vertex[above left of=a] (i1) {$e^-$};
    \vertex[below left of=a] (i2) {$e^-$};
    \vertex[above right of=b] (f1) {$\mu^-$};
    \vertex[below right of=b] (f2) {$\mu^+$};

    \vertex[below right=0.75cm of b] (x1);
    \vertex[below left=0.75cm of x1] (x2);

    \diagram* {
        (i1) --[fermion] (a) --[photon] (b) --[fermion] (f1),
        (i2) --[anti fermion] (a), (b) --[anti fermion] (f2),
        (x1) --[photon] (x2)
    };
\end{feynman}
\end{tikzpicture}
\end{center}


There are four fermion propagator self-energies, as well as two vertex corrections and a photon vacuum-polarization diagram:


\begin{center}
\begin{tikzpicture}
\begin{feynman}
    \vertex (a);
    \vertex[right of=a] (b);
    \vertex[above left of=a] (i1) {$e^-$};
    \vertex[below left of=a] (i2) {$e^-$};
    \vertex[above right of=b] (f1) {$\mu^-$};
    \vertex[below right of=b] (f2) {$\mu^+$};

    \vertex[above left=0.25cm of a] (x2);
    \vertex[above left=0.6cm of x2] (x1);

    \diagram* {
        (i1) --[fermion] (a) --[photon] (b) --[fermion] (f1),
        (i2) --[anti fermion] (a), (b) --[anti fermion] (f2),
        (x1) --[photon, half left] (x2)
    };
\end{feynman}
\end{tikzpicture}
\begin{tikzpicture}
\begin{feynman}
    \vertex (a);
    \vertex[right of=a] (b);
    \vertex[above left of=a] (i1) {$e^-$};
    \vertex[below left of=a] (i2) {$e^-$};
    \vertex[above right of=b] (f1) {$\mu^-$};
    \vertex[below right of=b] (f2) {$\mu^+$};

    \vertex[below left=0.25cm of a] (x2);
    \vertex[below left=0.6cm of x2] (x1);

    \diagram* {
        (i1) --[fermion] (a) --[photon] (b) --[fermion] (f1),
        (i2) --[anti fermion] (a), (b) --[anti fermion] (f2),
        (x1) --[photon, half right] (x2)
    };
\end{feynman}
\end{tikzpicture}

\begin{tikzpicture}
\begin{feynman}
    \vertex (a);
    \vertex[right of=a] (b);
    \vertex[above left of=a] (i1) {$e^-$};
    \vertex[below left of=a] (i2) {$e^-$};
    \vertex[above right of=b] (f1) {$\mu^-$};
    \vertex[below right of=b] (f2) {$\mu^+$};

    \vertex[above right=0.25cm of b] (x2);
    \vertex[above right=0.6cm of x2] (x1);

    \diagram* {
        (i1) --[fermion] (a) --[photon] (b) --[fermion] (f1),
        (i2) --[anti fermion] (a), (b) --[anti fermion] (f2),
        (x1) --[photon, half right] (x2)
    };
\end{feynman}
\end{tikzpicture}
\begin{tikzpicture}
\begin{feynman}
    \vertex (a);
    \vertex[right of=a] (b);
    \vertex[above left of=a] (i1) {$e^-$};
    \vertex[below left of=a] (i2) {$e^-$};
    \vertex[above right of=b] (f1) {$\mu^-$};
    \vertex[below right of=b] (f2) {$\mu^+$};

    \vertex[below right=0.25cm of b] (x2);
    \vertex[below right=0.6cm of x2] (x1);

    \diagram* {
        (i1) --[fermion] (a) --[photon] (b) --[fermion] (f1),
        (i2) --[anti fermion] (a), (b) --[anti fermion] (f2),
        (x1) --[photon, half left] (x2)
    };
\end{feynman}
\end{tikzpicture}
\end{center}


\begin{center}
\begin{tikzpicture}
\begin{feynman}
    \vertex (a);
    \vertex[right of=a] (b);
    \vertex[above left of=a] (i1) {$e^-$};
    \vertex[below left of=a] (i2) {$e^-$};
    \vertex[above right of=b] (f1) {$\mu^-$};
    \vertex[below right of=b] (f2) {$\mu^+$};

    \vertex[above left=0.75cm of a] (x1);
    \vertex[below left=0.75cm of a] (x2);

    \diagram* {
        (i1) --[fermion] (a) --[photon] (b) --[fermion] (f1),
        (i2) --[anti fermion] (a), (b) --[anti fermion] (f2),
        (x1) --[photon] (x2)
    };
\end{feynman}
\end{tikzpicture}
\begin{tikzpicture}
\begin{feynman}
    \vertex (a);
    \vertex[right of=a] (b);
    \vertex[above left of=a] (i1) {$e^-$};
    \vertex[below left of=a] (i2) {$e^-$};
    \vertex[above right of=b] (f1) {$\mu^-$};
    \vertex[below right of=b] (f2) {$\mu^+$};

    \vertex[above right=0.75cm of b] (x1);
    \vertex[below right=0.75cm of b] (x2);

    \diagram* {
        (i1) --[fermion] (a) --[photon] (b) --[fermion] (f1),
        (i2) --[anti fermion] (a), (b) --[anti fermion] (f2),
        (x1) --[photon] (x2)
    };
\end{feynman}
\end{tikzpicture}
\begin{tikzpicture}
\begin{feynman}
    \vertex (a);
    \vertex[right of=a] (x1);
    \vertex[right of=x1] (x2);
    \vertex[right of=x2] (b);
    \vertex[above left of=a] (i1) {$e^-$};
    \vertex[below left of=a] (i2) {$e^-$};
    \vertex[above right of=b] (f1) {$\mu^-$};
    \vertex[below right of=b] (f2) {$\mu^+$};


    \diagram* {
        (i1) --[fermion] (a) --[photon] (x1),
        (x2) --[photon] (b) --[fermion] (f1),
        (i2) --[anti fermion] (a), (b) --[anti fermion] (f2),
        (x1) --[fermion, half left] (x2) --[fermion, half left] (x1)
    };
\end{feynman}
\end{tikzpicture}
\end{center}



There are two ``box'' diagrams where either the electron or positron emits a virtual photon that is then absorbed by either the muon or anti-muon:


\begin{center}
\begin{tikzpicture}
\begin{feynman}
    \vertex (ul);
    \vertex[below of=ul] (bl);
    \vertex[right of=ul] (ur);
    \vertex[below of=ur] (br);

    \vertex[above left=of ul] (i1) {$e^-$};
    \vertex[below left=of bl] (i2) {$e^+$};
    \vertex[above right=of ur] (f1) {$\mu^-$};
    \vertex[below right=of br] (f2) {$\mu^+$};

    \diagram *{
        (i1) --[fermion] (ul) --[photon] (ur) --[fermion] (f1),
        (i2) --[anti fermion] (bl) --[photon] (br) --[anti fermion] (f2),
        (ul) --[fermion] (bl), (ur) --[fermion] (br)
    };
\end{feynman}
\end{tikzpicture}
\begin{tikzpicture}
\begin{feynman}
    \vertex (ul);
    \vertex[below of=ul] (bl);
    \vertex[right of=ul] (ur);
    \vertex[below of=ur] (br);

    \vertex[above left=of ul] (i1) {$e^-$};
    \vertex[below left=of bl] (i2) {$e^+$};
    \vertex[above right=of ur] (f1) {$\mu^-$};
    \vertex[below right=of br] (f2) {$\mu^+$};

    \diagram *{
        (i1) --[fermion] (ul) --[photon] (ur) --[anti fermion] (f2),
        (i2) --[anti fermion] (bl) --[photon] (br) --[fermion] (f1),
        (ul) --[fermion] (bl), (ur) --[fermion] (br)
    };
\end{feynman}
\end{tikzpicture}
\end{center}