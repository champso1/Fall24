\section{(18.5a)}

The momentum density is given by

\begin{equation}
    \pi^i = T^{0i} = \left[ \sum_n \diffp[]{\lag}{(\ddp_0 \psi_n)}\ddp^i \psi_n \right] - g^{0i}\lag,
\end{equation}

and the Dirac Lagrangian is given by

\begin{equation}
    \lag = i\psib\gamma^{\mu}\ddp_{\mu}\psi - m\psib\psi.
\end{equation}

For the brackets in the formula for the momentum density, we will have a sum over the normal field $\psi$ and the adjoint field $\psib$. Before doing either, it will be helpful to rewrite the Lagrangian and split the time and spatial components of the 4-gradient:

\begin{equation*}
    \lag = i\psib\gamma^0\ddp_0\psi + i\psib\gamma^i\ddp_i\psi - m\psib\psi.
\end{equation*}

Now, looking at $\psi$ first:

\begin{equation*}
    \diffp[]{\lag}{(\ddp_0 \psi)}\ddp^i \psi = i\psib\gamma^0\ddp^i\psi.
\end{equation*}

Now for $\psib$:

\begin{equation*}
    \diffp[]{\lag}{(\ddp_0 \psib)}\ddp^i \psib = 0,
\end{equation*}

since there are no derivatives of the adjoint field present in the Dirac Lagrangian. Hence,

\begin{equation*}
    T^{0i} = i\psib\gamma^0\ddp^i\psi - g^{0i}\left( i\psib\gamma^{\mu}\ddp_{\mu}\psi - m\psib\psi \right).
\end{equation*}

However, $g^{0i}$ is always zero, since $i$ is a spatial index and is never zero, and $g^{\mu\nu}=0$ when $\mu\neq\nu$. Hence,

\begin{equation*}
    T^{0i} = i\psib\gamma^0\ddp^i\psi.
\end{equation*}