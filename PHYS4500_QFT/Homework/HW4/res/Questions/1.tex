\section{(12.6)}


\begin{parts}
\item We are to argue that


\begin{equation}
    \hat{\phi}(\vv{x},0) \Ket{0} = \FourierInt{p} \frac{e^{-i\dotprodv{p}{x}}}{\sqrt{2p^0}} \Ket{\vv{p}}
\end{equation}

for a complex scalar field. The book gives a 1-particle Fock state with momentum $\vv{p}$ in terms of the creation operator on the vacuum state as

\begin{equation}
    \Ket{\vv{p}} = a^{\dagger}(\vv{p}) \Ket{0},
\end{equation}

with no normalization constant. Our convention is that

\begin{equation}
    \Ket{\vv{p}} = \sqrt{2p^0} a^{\dagger}(\vv{p}) \Ket{0},
\end{equation}

so I'll get the answer that the book wants then show what the answer would be with the normalization constant in our convention.

To evaluate the field operator on the vacuum state, let's write out $\hat{\phi}(\vv{x},0)$ in terms of the creation and annihilation operators:

\begin{equation*}
    \hat{\phi}(\vv{x},t) = \FourierIntE{p} \left[ a(\vv{p})e^{-i\dotprod{p}{x}} + b^{\dagger}(\vv{p})e^{i\dotprod{p}{x}} \right].
\end{equation*}

Now, at $t=0$,

\begin{equation*}
    e^{-i\dotprod{p}{x}} = e^{-ip^0x_0}e^{-ip^ix_i} = e^{-ip^0t}e^{i\dotprodv{p}{x}} \rightarrow e^{i\dotprodv{p}{x}},
\end{equation*}

so

\begin{equation*}
    \hat{\phi}(\vv{x},0) = \FourierIntE{p} \left[ a(\vv{p})e^{i\dotprodv{p}{x}} + a^{\dagger}(\vv{p})e^{-i\dotprodv{p}{x}} \right].
\end{equation*}

Letting this act on the vacuum state:

\begin{equation*}
    \hat{\phi}(\vv{x},0) \Ket{0} = \FourierIntE{p} \left[ a(\vv{p})e^{i\dotprodv{p}{x}} + b^{\dagger}(\vv{p})e^{-i\dotprodv{p}{x}} \right] \Ket{0}.
\end{equation*}

The first term will look like $a(\vv{p})\Ket{0}$, which is the annihilation operator acting on the vacuum state, which is zero, so all we have left is

\begin{equation*}
    \hat{\phi}(\vv{x},0) \Ket{0} = \FourierIntE{p} b^{\dagger}(\vv{p})e^{-i\dotprodv{p}{x}} \Ket{0} = \FourierIntE{p} e^{-i\dotprodv{p}{x}} b^{\dagger}(\vv{p}) \Ket{0},
\end{equation*}

but from the book's convention this is

\begin{equation*}
    \hat{\phi}(\vv{x},0) \Ket{0} = \FourierInt{p} \frac{e^{-i\dotprodv{p}{x}}}{\sqrt{2p^0}} \Ket{\vv{p}},
\end{equation*}

as expected. In our notation, we would need an extra factor of $\sqrt{2p^0}$ in the numerator and hence also the denominator, so we'd have:

\begin{equation*}
    \hat{\phi}(\vv{x},0) \Ket{0} = \FourierInt{p} \frac{e^{-i\dotprodv{p}{x}}}{2p^0} \Ket{\vv{p}}.
\end{equation*}






\item Qualitatively, since $\hat{\phi}(\vv{x},0)\Ket{0}$ (roughly) creates an anti-particle at position $\vv{x}$, the naive assumption would be that $\hat{\phi}^{\dagger}(\vv{x},0)\Ket{0}$ creates a particle at position $\vv{x}$. Expanding out the Hermitian conjugate of the complex scalar field in terms of creation and annihilation operators:

\begin{equation*}
    \hat{\phi}^{\dagger}(\vv{x},t) = \FourierIntE{p} \left[ b(\vv{p})e^{-i\dotprod{p}{x}} + a^{\dagger}(\vv{p})e^{i\dotprod{p}{x}} \right],
\end{equation*}

or, when $t=0$, 

\begin{equation*}
    \hat{\phi}^{\dagger}(\vv{x},t) = \FourierIntE{p} \left[ b(\vv{p})e^{i\dotprodv{p}{x}} + a^{\dagger}(\vv{p})e^{-i\dotprodv{p}{x}} \right].
\end{equation*}

Letting this act on the vacuum state, we get a similar case where the term $b(\vv{p})\Ket{0}=0$, since $b(\vv{p})$ is the anti-particle annihilation operator. Hence (in the book's convention),

\begin{equation*}
    \hat{\phi}^{\dagger}(\vv{x},t)\Ket{0} = \FourierIntE{p} a^{\dagger}(\vv{p})e^{-i\dotprodv{p}{x}} \Ket{0} = \FourierIntE{p} e^{-i\dotprodv{p}{x}} \Ket{\vv{p}}.
\end{equation*}

Since 

\begin{equation*}
    \FourierIntE{p} e^{-i\dotprodv{p}{x}} = \delta^3(\vv{x}),
\end{equation*}

we can make the interpretation that having this act on a 1-particle Fock state $\Ket{\vv{p}}$ gives a state of definite position $\Ket{\vv{x}}$, because the delta function delivers a spike there. However, as described in the problem in the book, we have an extra factor of $1/\sqrt{2p^0}$, so this is not a perfect spike at $\vv{x}$, and hence not technically a perfect state of position $\Ket{\vv{x}}$. But, the idea of the problem was to get a \textit{rough} idea of what the field operator does, and we have found that \textit{roughly}

\begin{equation*}
    \hat{\phi}^{\dagger}(\vv{x},t)\Ket{0} \sim \Ket{\vv{x}}.
\end{equation*}

\end{parts}