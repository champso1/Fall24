% also in png
\section{}

We are considering the $t$-channel process for singly producing a top quark like $u(p_1) + b(p_2) \rightarrow d(p_3) + t(p_4)$
\begin{figure}[ht]
  \centering

  \begin{tikzpicture}
    \begin{feynman}[large]
      \vertex (v1);
      \vertex[below of=v1] (v2);

      \vertex[above=4mm of v1] (x1);
      \vertex[below=4mm of v2] (x2);

      \vertex[left of=x1] (i1) {$u$};
      \vertex[left of=x2] (i2) {$b$};
      \vertex[right of=x1] (f1) {$d$};
      \vertex[right of=x2] (f2) {$t$};

      \diagram* {
        (i1) --[fermion, edge label=$p_1$] (v1) --[fermion, edge label=$p_3$] (f1),
        (v1) --[boson, edge label=$W$, momentum'=$q$] (v2),
        (i2) --[fermion, edge label'=$p_2$] (v2) --[fermion, edge label'=$p_4$] (f2)
      };
    \end{feynman}
  \end{tikzpicture}
  
  \caption{The Feynman diagram for the $t$-channel process $u(p_1) + b(p_2) \rightarrow d(p_3) + t(p_4)$.}
  \label{fig:1}
\end{figure}


Using our Feynman rules, we can pretty easily write down the amplitude:

\begin{equation}
  i\mathcal{M} = \bar{u}(p_3) \frac{(-ie)\gamma^\mu(1-\gamma^5)V_{ud}}{2\sqrt{2} \, \sin\theta_W} u(p_1) \frac{(-1) \br{g_{\mu\nu} - \frac{q_\mu q_\nu}{m_W^2}}}{q^2 - m_W^2 + i\epsilon} \bar{u}(p_4) \frac{(-ie) \gamma^\nu(1-\gamma^5)V_{bt}}{2\sqrt{2} \, \sin\theta_W} u(p_2).
\end{equation}
\begin{multline}
  \mathcal{M} = \frac{e^2 V_{ud}V_{bt}}{8\sin^2\theta_W [(p_1 - p_3)^2 - m_W^2]} \bar{u}(p_3)\gamma^\mu(1-\gamma^5)u(p_1) \\ \times \br{g_{\mu\nu} - \frac{(p_1 - p_3)_\mu(p_1 - p_3)\nu}{m_W^2}} \bar{u}(p_4)\gamma^\nu(1-\gamma^5)u(p_2)
\end{multline}

I believe that this is as far as we can get for just the amplitude. If we want to make a similar assumption as in the notes where we consider $q^2 << m_W^2$, then this becomes

\begin{equation}
  \mathcal{M} = -\frac{e^2 V_{ud}V_{bt}}{8\sin^2\theta_W \, m_W^2} \bar{u}(p_3)\gamma^\mu(1-\gamma^5)u(p_1) \bar{u}(p_4)\gamma_\mu(1-\gamma^5)u(p_2)
\end{equation}






%%% Local Variables:
%%% mode: LaTeX
%%% TeX-master: "../../HW13"
%%% End:
