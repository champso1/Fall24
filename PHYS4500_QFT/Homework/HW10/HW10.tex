\documentclass[titlepage]{article}
\usepackage{PreambleCommon,PreambleHW}


\title{HW10 \\[5pt] PHYS4500: Quantum Field Theory}
\author{Casey Hampson}

\begin{document}
\maketitle
\pagebreak



\section*{The Vacuum Polarization}

The vacuum polarization Feynman diagram is given by

\begin{center}
\begin{tikzpicture}
\begin{feynman}
    \vertex (a);
    \vertex[right of=a] (x1);
    \vertex[right of=x1] (x2);
    \vertex[right of=x2] (b);

    \diagram* {
        (a) --[photon, momentum'=$k$] (x1),
        (x2) --[photon, momentum'=$k$] (b),
        (x1) --[fermion, half left, edge label=$p$] (x2) --[fermion, half left, edge label=$p-k$] (x1)
    };
\end{feynman}
\end{tikzpicture}
\end{center}

With the Feynman rule that closed fermion loops get a factor of -1 and a trace, we have that the amplitude is

\begin{equation}
i\Pi^{\mu\nu}(k) = \int \frac{\dd^n p}{(2\pi)^n} (-1) \frac{\Tr[(-ie\gamma^{\mu})i(\psl + m)(-ie\gamma^{\nu})i(\psl - \ksl + m)]}{(p^2-m^2)[(p-k)^2-m^2]}.
\end{equation}

Simplifying a little bit and using Feynman parametrization on the denominator we get

\begin{equation}
    \Pi^{\mu\nu}(k) = ie^2 \int_0^1 \ddx \int \frac{\dd^n p}{(2\pi)^n} \frac{\Tr[\gamma^{\mu}(\psl + m)\gamma^{\nu}(\psl - \ksl + m)]}{[(p-k)^2x - m^2x + (p^2 - m^2)(1-x)]^2}
\end{equation}


Doing the trace, we know that the terms containing an odd number of gamma matrices will be zero, so

\begin{equation}
    \Tr[\gamma^{\mu}(\psl + m)\gamma^{\nu}(\psl - \ksl + m)] = \gamma^{\mu}\psl\gamma^{\nu}\psl - \gamma^{\mu}\psl\gamma^{\nu}\ksl + m^2\gamma^{\mu}\gamma^{\nu}.
\end{equation}

Looking at just the first term:

\begin{align}
    \gamma^{\mu}\psl\gamma^{\nu}\psl &= p_{\rho}p_{\gamma}\gamma^{\mu}\gamma^{\rho}\gamma^{\nu}\gamma^{\sigma} \\
    &= p_{\rho}p_{\sigma} \cdot 4\br{g^{\mu\rho}g^{\nu\sigma} - g^{\mu\nu}g^{\rho\sigma} + g^{\mu\sigma}g^{\rho\nu}} \\
    &= p^{\mu}p^{\nu} - g^{\mu\nu}p^2 - p^{\nu}p^{\mu} \\
    &= 2p^{\mu}p^{\nu} - g^{\mu\nu}p^2.
\end{align}

The other trace will be the same with the second $p$ being $k$ instead. We will also not be able to combine the first and last terms like I did in the final line above. The $m^2$ trace is simple, so in total we have

\begin{equation}
    \Tr[\gamma^{\mu}(\psl + m)\gamma^{\nu}(\psl - \ksl + m)] = 4[2p^{\mu}p^{\nu} - p^{\mu}k^{\nu} - k^{\mu}p^{\nu} - g^{\mu\nu}(p^2 - pk + m^2)],
\end{equation}

so

\begin{equation}
    \Pi^{\mu\nu}(k) = \frac{4ie^2}{(2\pi)^n} \int_0^1 \ddx \int \dd^np \frac{2p^{\mu}p^{\nu} - p^{\mu}k^{\nu} - k^{\mu}p^{\nu} - g^{\mu\nu}(p^2 - pk + m^2)}{[(p-k)^2x - m^2x + (p^2 - m^2)(1-x)]^2}.
\end{equation}

We can simplify the denominator such that we can introduce a shifted momentum:

\begin{gather}
    = p^2x - 2pkx - k^2x - m^2x + p^2 - m^2 - xp^2 + xm^2 \\
    = p^2 - 2pkx + k^2x - m^2 \\
    = p^2 - 2pkx + k^2x^2 - k^2x^2 + k^2x - m^2 \\
    = (p - xk)^2 + x(1-x)k^2 - m^2.
\end{gather}

With this, then we can define the shifted momentum $p' \equiv p-kx$ so that our denominator is now

\begin{equation}
    [(p')^2 + x(1-x)k^2 - m^2]^2.
\end{equation}

Now, to handle the terms in the numerator we have that $p = p' + kx$. Additionally, at this stage, we know that since the rest of the integrand is even, any terms linear in $p$ will integrate to zero since we are integrating over all possible momenta, which is a symmetric interval. Therefore,

\begin{align}
    p^{\mu}p^{\nu} &= p^{\mu\prime}p^{\nu\prime} + x^2k^{\mu}k^{\nu}, \\
    p^{\mu}k^{\nu} &= xk^{\mu}k^{\nu} \\
    k^{\mu}p^{\nu} &= xk^{\nu}k^{\mu} \\
    p^2 &= (p')^2 + x^2k^2, \\
    pk &= xk^2.
\end{align}

With all this, our amplitude is

\begin{equation}
    \Pi^{\mu\nu}(k) = \frac{4ie^2}{(2\pi)^n} \int_0^1 \ddx \int \dd^np' \frac{2p^{\mu\prime}p^{\nu\prime} - 2x(1-x)k^{\mu}k^{\nu} - g^{\mu\nu}[(p')^2 - x(1-x)k^2 + m^2]}{[(p') + x(1-x)k^2 - m^2]^2}.
\end{equation}

Rearranging to get all of the $p$ and $k$ bits separate:

\begin{equation}
    \Pi^{\mu\nu}(k) = \frac{4ie^2}{(2\pi)^n} \int_0^1 \ddx \int \dd^np' \frac{2p^{\mu\prime}p^{\nu\prime} - g^{\mu\nu}(p')^2 - 2x(1-x)k^{\mu}k^{\nu} + g^{\mu\nu}[x(1-x)k^2 - m^2]}{[(p') + x(1-x)k^2 - m^2]^2}.
\end{equation}

The integral identity we need to handle the $p$ terms is

\begin{multline}
    I(q) = \int \dd^np \frac{p^{\mu}p^{\nu}}{(p^2 + 2pq - m^2)^{\alpha}} = \frac{i\pi^{n/2}}{\Gamma(\alpha)}(-q^2-m^2)^{(n/2)-\alpha} \\ \times \br{q^{\mu}q^{\nu} \Gamma\br{\alpha - \frac{n}{2}} + \frac{1}{2}g^{\mu\nu}(-q^2-m^2)\Gamma\br{\alpha-1-\frac{n}{2}}}.
\end{multline}

Here, we don't have any dot product of our integration variable with some other momentum in the denominator, meaning we have that $q=0$. We also have $\alpha=2$, and $-m^2$ is just the rest of the stuff in our denominator. With this, $\Gamma(2) = 1$, so

\begin{gather}
    \int \dd^np \frac{p^{\mu\prime}p^{\nu\prime}}{[(p')^2 + x(1-x)k^2 - m^2]^2} = i\pi^{n/2}[x(1-x)k^2-m^2]^{(n/2)-2} \cdot \frac{1}{2}g^{\mu\nu}[x(1-x)k^2-m^2]\Gamma\br{1 - \frac{n}{2}} \\
    = \frac{i\pi^{n/2}}{2}g^{\mu\nu}[x(1-x)k^2-m^2]^{(n/2)-1}\Gamma\br{1 - \frac{n}{2}}.
\end{gather}

This is the first term in our numerator; the second has a $(p')^2$. Essentially, this just means the upper $\nu$ index goes to a lowered $\mu$ index, so we get $g^{\mu}_{\mu} = n$ in this integral - it's identical otherwise.

The other integral we need that carries no additional $p$ dependence in the numerator (with $q=0$) is:

\begin{equation}
    I(q) = \int \frac{\dd^np}{(p^2 + 2pq - m^2)^{\alpha}} = i\pi^{n/2} \frac{\Gamma\br{\alpha - \frac{n}{2}}}{\Gamma(\alpha)} (-m^2)^{(n/2) - \alpha}.
\end{equation}

Again, we have $\alpha=2$ meaning $\Gamma(2) = 1$ along with $-m^2$ being the rest of our denominator, so

\begin{equation}
    \int \frac{\dd^np}{[(p')^2 + x(1-x)k^2 -m^2]^2} = i\pi^{n/2}\Gamma\br{2 - \frac{n}{2}}[x(1-x)k^2 - m^2]^{(n/2)-2}.
\end{equation}

Putting all of this together, then, we get

\begin{multline}
    \Pi^{\mu\nu}(k) = -\frac{4e^2}{2^n \pi^{n/2}} \int_0^1 \ddx \big\{ g^{\mu\nu}\br{1 - \frac{n}{2}} [x(1-x)k^2 - m^2]^{(n/2)-1} \Gamma\br{1 - \frac{n}{2}} \\ + (-2x(1-x)k^{\mu}k^{\nu} + g^{\mu\nu}[x(1-x)k^2 - m^2]) \Gamma\br{2 - \frac{n}{2}}[x(1-x)k^2 - m^2]^{(n/2)-2} \big\}.
\end{multline}

Since $z\Gamma(z) = \Gamma(z+1)$, then

\begin{equation}
    \Gamma\br{2 - \frac{n}{2}} = \br{1 - \frac{n}{2}}\Gamma\br{1 - \frac{n}{2}},
\end{equation}

so, after some more simplification, we get

\begin{multline}
    \Pi^{\mu\nu}(k) = -\frac{4e^2}{2^n \pi^{n/2}} \br{1 - \frac{n}{2}}\Gamma\br{1 - \frac{n}{2}} \int_0^1 \ddx \; [x(1-x)k^2 - m^2]^{(n/2) - 2} \\ \times \left\{  g^{\mu\nu}[x(1-x)k^2 - m^2] - 2x(1-x)k^{\mu}k^{\nu} + g^{\mu\nu}[x(1-x)k^2 + m^2]  \right\},
\end{multline}
\begin{equation}
    \Pi^{\mu\nu}(k) = -\frac{8e^2}{2^n \pi^{n/2}} \br{1 - \frac{n}{2}}\Gamma\br{1 - \frac{n}{2}} \int_0^1 \ddx \; [x(1-x)k^2 - m^2]^{(n/2) - 2} \times x(1-x) (  g^{\mu\nu}k^2 - k^{\mu}k^{\nu} ).
\end{equation}

At this point, we can replace our $n$-dimensional integral with $n=4-\epsilon$ where we take $\epsilon\rightarrow0$:

\begin{equation}
    \Pi^{\mu\nu}(k) = -\frac{8e^2}{2^{4-\epsilon} \pi^{2-(\epsilon/2)}} \br{-1 + \frac{\epsilon}{2}}\Gamma\br{-1 + \frac{\epsilon}{2}} \int_0^1 \ddx \; [x(1-x)k^2 - m^2]^{-\epsilon/2} x(1-x) (  g^{\mu\nu}k^2 - k^{\mu}k^{\nu} ).
\end{equation}

Of course, everything of $\mathcal{O}(\epsilon)$ will just be zero as there are no divergences there. So, the term in brackets in the integrand is just 1, and we can make another simplification or two to arrive at

\begin{equation}
    \Pi^{\mu\nu}(k) = \frac{e^2}{2\pi^2} \Gamma\br{-1 + \frac{\epsilon}{2}} (g^{\mu\nu}k^2 - k^{\mu}k^{\nu}) \int_0^1 x(1-x) \;\ddx.
\end{equation}

We can expand the gamma function like so:

\begin{equation}
    \Gamma\br{-1 + \frac{\epsilon}{2}} = -\frac{2}{\epsilon} + \mathcal{O}(\epsilon^0),
\end{equation}

where we will keep the constant terms around bundled up in the $\mathcal{0}(\epsilon^0)$ since they are of course not zero, but we want to just isolate the divergent parts. Now, the integration over $x$ is easy:

\begin{equation}
    \int_0^1 x(1-x) \;\ddx = \left[ \frac{x^2}{2} - \frac{x^3}{3} \right]^1_0 = \frac{1}{2} - \frac{1}{3} = \frac{1}{6},
\end{equation}

so

\begin{equation}
    \boxed{\Pi^{\mu\nu}(k) = \frac{e^2}{6\pi^2\epsilon}(k^{\mu}k^{\nu} - g^{\mu\nu}k^2) + \mathcal{O}(\epsilon^0).}
\end{equation}


\end{document}
