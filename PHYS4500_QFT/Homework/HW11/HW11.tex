\documentclass[titlepage]{article}
\usepackage{PreambleCommon,PreambleHW}


\title{HW11 \\[5pt] PHYS4500: Quantum Field Theory}
\author{Casey Hampson}

\begin{document}
\maketitle
\pagebreak



From the lecture notes, we know already that

\begin{multline}
    \frac{\delta^2 Z_0[J]}{\delta J(x_1) \delta J(x_2)} = \left[ -D(x_1 - x_2) + \left( \int D(x-x_2) J(x) \;\dd^4x \right)\left( \int D(x-x_1) J(x) \;\dd^4x \right) \right] \\
    \times \exp\left( -\frac{1}{2}\int J(x) D(x-y) J(y) \;\dd^4x\dd^4y \right).
\end{multline}

If we take $x_2 \rightarrow x_3$ and $x_1 \rightarrow x_2$, then

\begin{multline}
    \frac{\delta^3 Z_0[J]}{\delta J(x_1) \delta J(x_2) \delta J(x_3)} = \frac{\delta}{\delta J(x_1)}\Bigg\{\left[ -D(x_2 - x_3) + \left( \int D(x-x_3) J(x) \;\dd^4x \right)\left( \int D(x-x_2) J(x) \;\dd^4x \right) \right] \\
    \times \exp\left( -\frac{1}{2}\int J(x) D(x-y) J(y) \;\dd^4x\dd^4y \right) \Bigg\}.
\end{multline}

Our goal is to show that

\begin{equation}
    G(x_1,x_2,x_3) = -i \frac{\delta^3 Z_0[J]}{\delta J(x_1) \delta J(x_2) \delta J(x_3)}\bigg|_{J=0} = 0.
\end{equation}

We have a product of two terms; the first is the expression in brackets, the second is the exponential. We will need to take derivatives of both. For notational simplicity, I will let

\begin{equation}
    Y_1(x_1,x_2) \equiv  -D(x_2 - x_3) + \left( \int D(x-x_3) J(x) \;\dd^4x \right)\left( \int D(x-x_2) J(x) \;\dd^4x \right),
\end{equation}

which is just the expression in brackets, and

\begin{equation}
    Y_2 \equiv \exp\left( -\frac{1}{2}\int J(x) D(x-y) J(y) \;\dd^4x\dd^4y \right).
\end{equation}

Then

\begin{equation}
    \frac{\delta^3 Z_0[J]}{\delta J(x_1) \delta J(x_2) \delta J(x_3)} = Y_2 \frac{\delta Y_1}{\delta J(x_1)} + Y_1 \frac{\delta Y_2}{\delta J(x_1)}.\label{eq:1}
\end{equation}

Looking at the first functional derivative:

\begin{align}
    \frac{\delta Y_1}{\delta J(x_1)} &= \frac{\delta}{\delta J(x_1)} \left[ -D(x_2 - x_3) + \left( \int D(x-x_3) J(x) \;\dd^4x \right)\left( \int D(x-x_2) J(x) \;\dd^4x \right) \right] \\
    &= -D(x_1 - x_3)\int D(x-x_2) J(x) \;\dd^4x - D(x_1 - x_2)\int D(x-x_3) J(x) \;\dd^4x.
\end{align}

Our goal is to, at the end, evaluate this quantity at $J=0$. Since every term here contains a $J$, this will be zero. Further, the entire first term in Equation~\eqref{eq:1} is zero. Looking next at the second functional derivative:

\begin{align}
    \frac{\delta Y_2}{\delta J(x_1)} &= \frac{\delta}{\delta J(x_1)} \exp\left( -\frac{1}{2}\int J(x)D(x-y)J(y) \dd^4x \right) \\
    &= \left( -\frac{1}{2}D(x-x_1)J(x)\;\dd^4x \right) \exp\left( -\frac{1}{2}\int J(x)D(x-y)J(y) \dd^4x \right).
\end{align}

There is really only one term here, and it contains a $J$, which, after evaluating it to zero, turns the entire expression zero. Therefore,

\begin{equation}
    G(x_1,x_2,x_3) = -i \frac{\delta^3 Z_0[J]}{\delta J(x_1) \delta J(x_2) \delta J(x_3)}\bigg|_{J=0} = 0,
\end{equation}

as we expect.

\end{document}
